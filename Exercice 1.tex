\documentclass[11pt,a4paper]{article}
\usepackage[utf8]{inputenc}
\usepackage[T1]{fontenc}
\usepackage[a4paper]{geometry}

\usepackage{amsthm}
\newtheorem{theo}{Theorem}
\newtheorem{defi}{Definition}
\newtheorem{prop}{Proposition}
\newtheorem{lem}{Lemma}

\usepackage{amssymb}
\usepackage{amsmath}
\usepackage{bbm}
\usepackage{stmaryrd}
\usepackage{proof}
\usepackage{tikz}
\usetikzlibrary{matrix}
\usetikzlibrary{decorations.pathmorphing}

\setlength\parindent{0pt}

\newcommand{\La}{\mathcal{L}}
\newcommand{\M}{\mathcal{M}}
\newcommand{\ph}{\varphi}
\newcommand{\itemz}{\item[$\triangleright$]}
\newcommand{\F}{\mathcal{F}}
\newcommand{\gr}{\textbf}
\newcommand{\il}{\textit}
\newcommand{\N}{\mathbb{N}}
\newcommand{\U}{\mathcal{U}}
\newcommand{\preuve}{\begin{proof}[Preuve]}
\newcommand{\cqfd}{\end{proof}}
\newcommand{\equi}{\Leftrightarrow}
\newcommand{\R}{\mathbb{R}}
\newcommand{\C}{\mathcal{C}}
\newcommand{\I}{\mathcal{I}}
\renewcommand{\iff}{\Leftrightarrow}
\newcommand{\T}{\mathcal{T}}
\newcommand{\V}{\mathcal{V}}
\newcommand{\lb}{\llbracket}
\newcommand{\rb}{\rrbracket}
\newcommand{\info}[1]{\text{{\fontfamily{lmss}\selectfont #1}}}
\newcommand{\Mod}{\info{Mod}}
\newcommand{\Sen}{\info{Sen}}
\newcommand{\1}{\mathbbm{1}}
\renewcommand{\P}{\mathcal{P}}
\newcommand{\D}{\mathcal{D}}

\usepackage[bitstream-charter]{mathdesign}
%\usepackage[charter]{mathdesign}
%\usepackage{fontspec}
%\setmainfont{URW Palladio L}

\begin{document}

\section*{Problem 2.1.2}

Assume $A \cap B \neq \emptyset$. Let $c \in A \cap B$. Then $0 \leq z(A,B) = \inf_{(a,b) \in A \times B} d(a,b) \leq d(c,c) = 0$ so that $z(A,B) = 0$. The converse is not always true: if $X = \R$ and $d(x,y) = |x-y|$, take $A = \{0\}$ and $B = \R_+^*$. Then $A \cap B = 0$ but for all $\varepsilon > 0$ we have $\varepsilon \in B$ so $z(A,B) = d(0,B) \leq d(0,\varepsilon) = \varepsilon$. Thus $z(A,B) = 0$.\\\\
Concerning the second question, first note that if $X = \R$ and $d(x,y) = |x-y|$, taking $A = \{0\}$ and $B = \R_+^*$ yields $\sup_{b \in B} \inf_{a\in A} d(a,b) = sup_{\varepsilon > 0} \varepsilon = +\infty$. For the function $d_H$ to be finite, we have thus to impose further that we are considering only \il{bounded} closed non-empty subsets of $X$. 
\begin{defi}[Bounded subset] Let $A \in \P(X)$. $A$ is bounded if its diameter $\delta(A)$, defined by $\delta(A) = \sup_{(a,a') \in A \times A} d(a,a')$, is finite.
\end{defi}
For example, every subset is automatically bounded if $d$ is bounded.\\\\
Let $E \subseteq \P(X)$ be the space of bounded closed non-empty subsets of $X$. Then the function $d_H : E \times E \to \R$ is well-defined. Indeed, let $A,B \in E$. Take $a_0 \in A$ and $b_0 \in B$. Then
\[ \forall (a,b) \in A \times B,  d(a,b) \leq d(a,a_0) +  d(a_0,b_0) + d(b_0,b) \leq \delta(A) + d(a_0,b_0) + \delta(B) \]
Then $d_H(A,B) \leq \delta(A) + d(a_0,b_0) + \delta(B) < +\infty$. Let us show that $d_H$ is a distance.
Let $A,B,C \in E$. First, remark that $d_H(A,B) = \max\{ \sup_{a \in A} d(a,B) , \sup_{b \in B} d(b,A) \}$.
\begin{itemize}
\setlength\itemsep{-0.3em}
\item[(i)] As $d \geq 0$, it is immediate that $d_H \geq 0$.
\item[(ii)] If $A = B$ then $\sup_{a\in A} d(a,B) = \sup_{a \in A} 0 = 0$ because $A \subseteq B$. Similarly, $\sup_{b\in B} d(b,A) = 0$. Thus $d_H(A,B) = 0$.
\item[] If $d_H(A,B) = 0$ then $\sup_{a\in A} d(a,B) = 0$ so for any $a \in A$ we have $d(a,B) = 0$. As $B$ is closed, this means that $a \in B$ (according to remark 2.1.2). Thus $A \subseteq B$. Similarly $B \subseteq A$, and finally $A = B$.
\item[(iii)] The function $d_H : E \times E \to \R$ is symmetric because the function $\max : \R \times \R \to \R$ is symmetric.
\item[(iv)] Let $a \in A$, $b\in B$, $c\in C$. Then $d(a,b) \leq d(a,c) + d(c,b)$. Because $d(a,c)$ does not depend on $b$, taking the infimum for $b \in B$ yields $d(a,B) \leq d(a,c) + d(c,B)$. As $d(c,B) \leq d_H(C,B)$ we get $d(a,B) \leq d(a,c) + d_H(C,B)$. Taking the infimum for $c \in C$ yields $d(a,B) \leq d(a,C) + d_H(C,B)$. As $d(a,C) \leq d_H(A,C)$ we get $d(a,B) \leq d_H(A,C) + d_H(C,B)$, thus $\sup_{a\in A} d(a,B) \leq d_H(A,C) + d_H(C,B)$. With the same methode one can prove that $\sup_{b\in B} d(b,A) \leq d_H(A,C) + d_H(C,B)$. Thus, $d_H(A,B) \leq d_H(A,C) + d_H(C,B)$.
\end{itemize}
\newpage
\section*{Problem 4.7.2}
$(i) \Rightarrow (ii)$. Assume $A$ is hermitian. Then $f_A(x,y) = \langle Ax, y \rangle = \langle x, Ay \rangle = \overline{\langle Ay , x \rangle} = \overline{ f_A(y,x) }$.\\\\
$(ii) \Rightarrow (iii)$. Assume $f_A$ is hermitian. Then $\phi_A(x) = \langle Ax, x \rangle = \overline{ \langle Ax , x \rangle}$ so $\langle Ax , x \rangle \in \R$.\\\\
$(iii) \Rightarrow (i)$. Assume $\phi_A$ is real. Then for any $x,y \in H$:
\[ \langle Ax , y \rangle + \langle Ay , x \rangle = \phi_A(x+y) - \phi_A(x) - \phi_A(y) \in \R \]
So we have the following equalities
\begin{align*}
\langle Ax , y \rangle + \langle Ay , x \rangle & = \overline{\langle Ax , y \rangle} + \overline{\langle Ay , x \rangle} \\ \langle x , A^* y \rangle + \overline{\langle x , A y \rangle} & = \overline{\langle x , A^* y \rangle} + \langle x , Ay \rangle \\
\langle x , (A^* - A) y \rangle & = \overline{\langle x , (A^*-A) y \rangle}
\end{align*}
So $\langle x , (A^* - A) y \rangle \in \R$. Once applied to the couple $(ix,y)$ this yields $i \langle x , (A^* - A)y \rangle \in \R$. If $\langle x , (A^* - A) \rangle \neq 0$ then $i \in \R$, contradiction. Thus
\[ \forall x,y \in H, \langle x, (A-A^*)y \rangle = 0 \]
Fix $y$ and define the linear continuous function $f(x) = \langle x , (A-A^*)y \rangle$. By the Riesz representation theorem, we have $||(A-A^*)y||_H = ||f||_{H^*} = 0$ because $f = 0$. Hence $(A-A^*)y = 0$. This is true for all $y \in H$, so $A = A^*$. This proves that $A$ is hermitian.\\\\
Assume that $A$ is hermitian. We refer to definition 3.2.2, and also define $[A]$ as follows:
\begin{align*} & ||A|| = \sup_{x \neq 0} \frac{||Ax||}{||x||} & [A] = \sup_{x \neq 0} \frac{|\langle Ax , x \rangle|}{||x||^2} \end{align*}
Cauchy-Schwarz gives $|\langle Ax , x \rangle| \leq ||Ax|| \cdot ||x||$, thus if $x \neq 0$ we have $|\langle Ax , x \rangle|/||x||^2 \leq ||Ax||/||x||$, so that $[A] \leq ||A||$. To show the other inequality, it is sufficient to see that for all non-zero $x,y \in H$, $|\langle Ax , y \rangle| \leq [A] \cdot ||x||\cdot||y||$, because if this is the case, taking $y = Ax/||Ax||$ yields $||Ax|| \leq [A]\cdot ||x||$ and thus $||A|| \leq [A]$.\\\\
The inequality $|\langle Ax , y \rangle| \leq [A] \cdot ||x|| \cdot ||y||$ is unchanged by every transformation $y \mapsto e^{i\theta} y$, so we can assume that $\langle Ax , y \rangle \in \R$. In this case, we have $\langle Ax , y \rangle = \langle x , Ay \rangle = \langle Ay , x \rangle$, so:
\begin{align*} & \langle Ax , y \rangle = \frac{\langle A(x+y) , x+y \rangle - \langle A(x-y) , x-y \rangle}{4} \\
& |\langle Ax , y \rangle| \leq \frac{[A]\left(||x+y||^2 + ||x-y||^2 \right)}{4} = \frac{[A]\left(||x||^2 + ||y||^2 \right)}{2}
\end{align*}
by the parallelogram identity. Take $x \mapsto \sqrt{\frac{||y||}{||x||}} x$, $y \mapsto \sqrt{\frac{||x||}{||y||}} y$ to get the  desired inequality:
\[ |\langle Ax , y \rangle| \leq [A] \cdot ||x|| \cdot ||y|| \]
This completes the proof that $[A] = ||A||$.
\newpage
\section*{Exercise.}
\il{Let $X$ be a Banach space. Prove that if a sequence $(x_n)_{n\in\N}$ is a relatively compact set and converges weakly in $X$, then $(x_n)_{n\in\N}$ converges strongly in $X$.}\\\\
Let $x \in X$ be the weak limit of $(x_n)_{n\in\N}$. If $x_n \not\to x$ then it is possible to find an $\varepsilon > 0$ such that
\[ \forall n \in \N , \exists k \geq n , ||x_k-x|| \geq \varepsilon \]
so we can extract a subsequence $(x_{\ph(n)})_{n\in\N}$ such that for all $n \in \N$, $||x_{\ph(n)} - x|| \geq \varepsilon$. The sequence $(x_{\ph(n)})_{n\in\N}$ is contained in the compact set $\overline{\{x_n\mid n \in \N\}}$, so it has a converging subsequence $x_{(\ph \circ \psi)(n)}\to y \in X$. Strong convergence implies weak convergence, so $x_{(\ph \circ \psi)(n)} \rightharpoonup y$. Moreover, $x_n \rightharpoonup x$ so we have $x_{(\ph \circ \psi)(n)} \rightharpoonup x$. Uniqueness of the weak limit yields $y = x$. Hence $x_{(\ph \circ \psi)(n)} \to x$. But by definition of $\ph$, for every $n \in \N$ we have $||x_{(\ph\circ\psi)(n)} - x|| > \varepsilon$. There is a contradiction. Thus $x_n \to x$.
\section*{Exercise.}
\il{Let $X,Y$ be Banach spaces. Let $A \in \mathcal{K}(X,Y)$. Prove that if $x_n \rightharpoonup x \in X$ then $Ax_n \to Ax \in Y$.}\\\\
Assume $x_n \rightharpoonup x \in X$. Then $\{x_n \mid n\in\N \}$ is bounded. As $A$ is a compact operator, $\{ Ax_n \mid n\in\N\}$ is relatively compact.\\\\
Let $\phi \in \La(Y)$, then $\phi \circ A \in \La(X)$. As $x_n \rightharpoonup x$, we have $(\phi \circ A)(x_n) \to (\phi \circ A)(x)$ i.e. $\phi(Ax_n) \to \phi(Ax)$. This proves that $Ax_n \rightharpoonup Ax$.\\\\
The sequence $(Ax_n)_{n\in\N}$ is a relatively compact set and converges weakly (towards $Ax$) in $Y$. According to the preceding exercise, $Ax_n \to Ax$.\newpage

\section*{Proposition 7.2.1.}
Let $f(x) \in \mathcal{D}'(\R^n)$ and $g(y) \in \mathcal{D}'(\R^m)$. In the following, $x$ implicitly belongs to the first space $\R^n$, and $y$ implicitly belongs to the second space $\R^m$.
\subsection*{(1)}
\begin{defi}
Let $\D(\R^n) \otimes \D(\R^m) \subseteq \D(\R^{n+m})$ be the space of test functions of the form $\ph(x,y) = \psi(x)\theta(y)$, where $\psi \in \D(\R^n)$ and $\theta \in \D(\R^m)$.
\end{defi}
\begin{lem}
The space $\textnormal{Span}(\D(\R^n) \otimes \D(\R^m))$ is dense in $\D(\R^{n+m})$.
\end{lem}
Let us prove the commutativity for all $\ph(x,y) = \sum_{i=1}^N \psi_i(x)\theta_i(y) \in \textnormal{Span}(\D(\R^n) \otimes \D(\R^m))$.
\begin{align*}
\langle f(x) \times g(y) , \psi(x)\theta(y) \rangle
\end{align*}
\subsection*{(4)}
Let $\alpha \in \N^{n+m}$ be a multi-index such that its last $m$ components are $0$. Then it can be viewed as a multi-index in $\N^n$, and we can write $D^\alpha_{(x,y)} = D^\alpha_x$. It is then easy to see that for any $\ph \in \mathcal{D}(\R^{n+m})$
\begin{align*}
\langle D^\alpha_x (f(x) \times g(y)) , \ph \rangle
& = (-1)^{|\alpha|} \langle f(x) \times g(y) , D^\alpha_x \ph \rangle \\
& = (-1)^{|\alpha|} \langle g(y) \times f(x) , D^\alpha_x \ph \rangle \\
& = (-1)^{|\alpha|} \langle g(y) , \langle f(x) , D^\alpha_x \ph \rangle \rangle \\
& = \langle g(y) , \langle D^\alpha_x f(x) , \ph \rangle \rangle \\
& = \langle g(y) \times D^\alpha_x f(x) , \ph \rangle \\
& = \langle D^\alpha_x f(x) \times g(y) , \ph \rangle 
\end{align*} 
This yields $D^\alpha_x (f(x) \times g(y)) = D^\alpha_x f(x) \times g(y)$.
\end{document}