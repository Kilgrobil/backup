\documentclass[11pt,a4paper]{report}
\usepackage[utf8]{inputenc}
\usepackage[T1]{fontenc}
\usepackage[a4paper]{geometry}

\usepackage{amsthm}
\newtheorem{theo}{Théorème}
\newtheorem{defi}[theo]{Définition}
\newtheorem{prop}[theo]{Proposition}

\usepackage{amssymb}
\usepackage{amsmath}
\usepackage{bbm}
\usepackage{stmaryrd}
\usepackage{proof}
\usepackage{tikz}
\usetikzlibrary{matrix}
\usetikzlibrary{decorations.pathmorphing}

\setlength\parindent{0pt}

\newcommand{\La}{\mathcal{L}}
\newcommand{\M}{\mathcal{M}}
\newcommand{\ph}{\varphi}
\newcommand{\itemz}{\item[$\triangleright$]}
\newcommand{\F}{\mathcal{F}}
\newcommand{\gr}{\textbf}
\newcommand{\il}{\textit}
\newcommand{\N}{\mathbb{N}}
\newcommand{\U}{\mathcal{U}}
\newcommand{\preuve}{\begin{proof}[Preuve]}
\newcommand{\cqfd}{\end{proof}}
\newcommand{\equi}{\Leftrightarrow}
\newcommand{\R}{\mathcal{R}}
\newcommand{\C}{\mathcal{C}}
\newcommand{\I}{\mathcal{I}}
\renewcommand{\iff}{\Leftrightarrow}
\newcommand{\T}{\mathcal{T}}
\newcommand{\V}{\mathcal{V}}
\newcommand{\lb}{\llbracket}
\newcommand{\rb}{\rrbracket}
\newcommand{\info}[1]{\text{{\fontfamily{lmss}\selectfont #1}}}
\newcommand{\Mod}{\info{Mod}}
\newcommand{\Sen}{\info{Sen}}
\newcommand{\1}{\mathbbm{1}}
\renewcommand{\contentsname}{Table des matières}
\renewcommand{\proofname}{Preuve}


\usepackage[bitstream-charter]{mathdesign}
%\usepackage[charter]{mathdesign}
%\usepackage{fontspec}
%\setmainfont{URW Palladio L}
\renewcommand\bibname{Bibliographie}



\begin{document}

Ce projet de recherche de troisième année à CentraleSupélec cherche à rendre compte du lien qui est en train d'être établi entre morphologie mathématique et théorie des institutions stratifiées.\\\\
La \gr{morphologie mathématique} est une discipline issue historiquement du traitement des images. Elle a été introduite par G. Matheron et J. Serra en 1964. Il s'agit d'une théorie d'analyse de structures munie d'aspects algébriques et topologiques importants. Elle manipule des objets concrets et a de nombreux champs d'application telles que l'imagerie médicale ou la géologie.\\\\
La \gr{théorie des institutions} est une formalisation de la logique proposée en 1992 par J. Goguen et R. Burstall \cite{Gog92}. Cette théorie a été étudiée systématiquement par R. Diaconescu \cite{Dia08}, et enrichie en 2007 par R. Diaconescu et M. Aiguier en la théorie des institutions \gr{stratifiées} \cite{Aig07} qui permet une plus grande expressivité.\\\\
Récemment, I. Bloch et M. Aiguier ont remarqué qu'un lien était à établir entre ces deux théories.


\newpage
\nocite{*}
\bibliographystyle{plain}
\bibliography{biblireport}

\end{document}